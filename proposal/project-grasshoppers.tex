%%%%%%%%%%%%%%%%%%%%%%%%%%%%%%%%%%%%%%%%%%%%%%%%%%%%%%%%%%%%
%%%%%%%%%%%%%%%%%%%%%%%%%%%%%%%%%%%%%%%%%%%%%%%%%%%%%%%%%%%%
\documentclass[paper=a4, fontsize=11pt]{scrartcl} 
\usepackage[left=3cm,right=2.5cm,top=2.5cm,bottom=3.5cm]{geometry}
\usepackage{graphicx}
\usepackage{url}
\usepackage{enumitem}

\setlength\parindent{0pt} 
\newcommand{\horrule}[1]{\rule{\linewidth}{#1}} 
\pagestyle{empty}
\begin{document}
\includegraphics[height=1.0cm]{lascropped}%
\hfill%
\raisebox{0.2cm}{\includegraphics[height=0.6cm]{ethcropped}}
{
\centering
\normalfont \normalsize 
\horrule{0.5pt} \\[0.2cm] 
\huge Automated recording system to monitor grasshopper abundance in natural environments
\\[0.3cm]
\normalfont \normalsize 
\textsc{Project Proposal for Master Thesis}
\horrule{2pt} \\[0.6cm] 
%\\[0.6cm]
}
%%%%%%%%%%%%%%%%%%%%%%%%%%%%%%%%%%%%%%%%%%%%%%%%%%%%%%%%%%%%
%%%%%%%%%%%%%%%%%%%%%%%%%%%%%%%%%%%%%%%%%%%%%%%%%%%%%%%%%%%%
%\vspace{-10mm}
\section*{Motivating Applications}
%\vspace{-3mm}

On-going climate change induces shifts in the distribution of species along latitude and elevation gradients, while land-use intensification is degrading the quality of habitats for biodiversity. Global changes are thus largely threatening existing semi-natural ecosystems. Monitoring data are required to design management strategies and limit the negative effect of global change on biodiversity. To measure the quality of the environment in many locations across the landscape, crowd-sourced monitoring tools have been developed, but currently those are mostly available for abiotic measurements (e.g. pollution). A device which provides an automated identification of species in the landscape would allow to delegate the collection of data to citizen to gather a much larger amount of data continuously. Currently, only a few taxonomic experts are mandated to collect biodiversity data, which is costly and the data availability is spatially and temporally limited. A species identification system in a form of an application for smartphone would allow to delegate data collection to citizen. Ultimately, this data will increase the knowledge of ecosystem response to environmental changes, since grasshoppers are indicators of ecosystem quality.

%%%%%%%%%%%%%%%%%%%%%%%%%%%%%%%%%%%%%%%%%%%%%%%%%%%%%%%%%%%%
%%%%%%%%%%%%%%%%%%%%%%%%%%%%%%%%%%%%%%%%%%%%%%%%%%%%%%%%%%%%
%\vspace{-4mm}
\section*{Scope of the Project}

The first goal is to develop an algorithm that outperforms a previous algorithm developed by William Ducret, described in a Bachelor Thesis \cite{2015-ducret}, by exploring alternative feature and machine learning methods. In a second step the algorithm should be improved to be robust against noise, such as human voices, cars, or bird songs. The algorithm should reliably detect grasshopper sound from smartphone recordings that are taken in a natural environment. During the first part of the project, the work will be targeted on available data from Switzerland for 20 species with more than 10 individual replicates per species, complemented by the DORSA database \cite{DORSA} which contains data for more than 700 European species. From the DORSA database, we will give a stronger emphasis on the species occurring in Switzerland. The algorithm should be able to reliably classify the species which are common in Switzerland. We will evaluate the usefulness of complementary prior to be included in the algorithm, such as the GPS location of the phone that is recording the sound in complement with the known distribution range of the species across Switzerland. In a third step, the algorithm will be integrated into a smartphone application. We will either build an independent application or if we will extend the existing Orthoptera-App\cite{orthoapp}. The goal of the app is to build a tool for crowd-sourcing the data gathering about grasshopper distribution and dynamics.


%%%%%%%%%%%%%%%%%%%%%%%%%%%%%%%%%%%%%%%%%%%%%%%%%%%%%%%%%%%%
%%%%%%%%%%%%%%%%%%%%%%%%%%%%%%%%%%%%%%%%%%%%%%%%%%%%%%%%%%%%
%\vspace{-4mm}
\section*{Milestones}
%\vspace{-3mm}

\begin{itemize}
  	\item Develop the algorithm that outperforms the Ducret algorithm on a preprocessed dataset(4 weeks)
	\item Collect a dataset that allows us to train a more robust algorithm (possibly DORSA) (1 week)
	\item Develop a robust version of the algorithm and test it against the data (7 weeks)
  	\item Implement the smartphone application (10 weeks)
  	\item Written thesis (4 weeks)
\end{itemize}

%%%%%%%%%%%%%%%%%%%%%%%%%%%%%%%%%%%%%%%%%%%%%%%%%%%%%%%%%%%%
%%%%%%%%%%%%%%%%%%%%%%%%%%%%%%%%%%%%%%%%%%%%%%%%%%%%%%%%%%%%
%\clearpage
\vspace{-4mm}
{
%\fontsize{11pt}{9pt}
\fontsize{10.5pt}{8pt}
\selectfont
\bibliographystyle{apalike}
\bibliography{project-grasshoppers}
}
%%%%%%%%%%%%%%%%%%%%%%%%%%%%%%%%%%%%%%%%%%%%%%%%%%%%%%%%%%%%
%%%%%%%%%%%%%%%%%%%%%%%%%%%%%%%%%%%%%%%%%%%%%%%%%%%%%%%%%%%%
\end{document}
%%%%%%%%%%%%%%%%%%%%%%%%%%%%%%%%%%%%%%%%%%%%%%%%%%%%%%%%%%%%
%%%%%%%%%%%%%%%%%%%%%%%%%%%%%%%%%%%%%%%%%%%%%%%%%%%%%%%%%%%%
%IMAGE Source
%https://commons.wikimedia.org/wiki/File:Gradient_descent.svg